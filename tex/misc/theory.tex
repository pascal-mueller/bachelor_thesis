% !TEX options=--shell-escape
\title{Zusammenfasung - Physik 1}
\author{
        Pascal Müller [pamuelle@student.ethz.ch]
}
\date{\today}

\documentclass[12pt]{article}
\usepackage[utf8]{inputenc}
\usepackage[parfill]{parskip}
\usepackage{amsmath}
\usepackage{hyperref}

\title{Theory}

\begin{document}
\maketitle

\section{Signal Processing}
\paragraph{Matched Filtering} Detect the presence of a template in the
unknown signal. Want to maximize SNR. Basically inner product.

\paragraph{SNR}
\[
  \text{SNR} = \frac{P_{signal}}{P_{noise}} 
  \Rightarrow \text{SNR} = \frac{E[S^2]}{E[N^2]}
\]


\section{Bayesian Statistics}
\paragraph{Parameters} are treated as random variables that can be described with
probability distributions. Often denoted by $\theta$

\paragraph{Probability} is our degree of belief.

\paragraph{Prior Distribution for $\theta$}  A prior distribution is a
\textbf{mathematical expression of our belief} about the distribution of the
parameter. E.g. the prior distribution for $\theta$ might be uniform.

\paragraph{Likelihood Functions} Probability distributions quantify the
probability of the data for a given parameter value (that is, $P(y|\theta))$,
whereas a likelihood function quantifies the likelihood of a parameter value
given the data (that is, $L(\theta|y))$. The functional form is the same for
both, and the notation is often used interchangeably (that is, $P(y|\theta) =
L(\theta|y))$. E.g. likelihood function might be binomial.

\paragraph{Posterior Distribution} Posterior = Prior * Likelihood
\[
  P(\theta | y) = P(\theta) * P(y | \theta)
\]

\section{Bayesian Inference}
\paragraph{Bayes Rule}
\[
  P(\theta | x) = \frac{P(x|\theta) \cdot P(\theta)}{P(x) }
\]

\end{document}
