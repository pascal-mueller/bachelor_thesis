% !TEX options=--shell-escape
\title{Bachelor Thesis}
\author{
        Pascal Müller [pamuelle@student.ethz.ch]
}
\date{\today}

\documentclass[12pt]{article}
\usepackage[utf8]{inputenc}
\usepackage[parfill]{parskip}
\usepackage{amsmath}
\usepackage{hyperref}

\begin{document}
\maketitle

\section{Introduction}
\section{Matched Filtering}
\section{Neural Network}
\subsection{Data Generation}
\subsubsection{Test Data}
\subsubsection{Training Data}
\subsubsection{Network}



\section{Introduction}
Some introductionary text: General idea is to first set the stage by describing
the landscape of GW detectors GW physics as a field of research. Followed by an
more in-depth description of the detectors. While I only use LIGO (H1, L1) I
also briefly describe Virgo. Next I describe the theory behind signal processing
and maybe parameter estimation. Next I describe my deep learning approach wher I
use the previous chapter(s) to make desicions on my model. Next I present my
  results.

Describe:
\begin{itemize}
  \item Set-Up of this document
  \item General setting
  \item Detectors: LIGO \& Virgo
  \item Observation Runs?
  \item Describe use of segments?
\end{itemize}


\section{LIGO-Virgo Basics}
\subsection{Detector properties}
A calibration prodecdure is applied to the interferometer photodiode output of
each detectors to produces GW strain data as a time series, sampled at 16384 HZ
for LIGO data and 20 kHz for Virgo data.

For Advanced LIGO detectors, the calibartion is valid above 10 Hz and below 5
kHz. For Advanced Virgo in O2 the calibration validity range was from 10 Hz to
8 kHz.

\subsection{Data Processiong}
- Describe pipeline for data processing (see ligo guide graph p. 13)

\subsection{


\end{document}
