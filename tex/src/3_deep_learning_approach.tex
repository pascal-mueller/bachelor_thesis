%!TEX root = ../thesis.tex

\section{Deep Learning Approach}
% Introduction: High-level overview about general idea behind DL.
% Main Workflow: Describe the main workflow (train + eval in each epoch)
% Training: Describe training
% Evaluation: Describe evaulation (incl. accuracy and efficiency)

% Quick introduction to deep learning
Deep learning is a machine learning approach which consists of processing
units, so called neurons, which are arranged in an array. Such an array makes up
a layer. One to several such layers make up a neural network (NN). Each neuron
acts like a filter extracting higher-level features from the raw input. On a 
basic level, a neural network is trained by repeatedly applying the training
data and comparing its results with the known labels.

% Extend on the introduction
For the DL approach one needs a training set and a test set. The training set
is split whereas 80\% is used for the actual trianing of the NN and 20\% is used
to evaluate the trained NN in each epoch. 

%!TEX root = ../thesis.tex

% Define block styles
\tikzstyle{decision} = [diamond, draw, fill=blue!20, 
    text width=4.5em, text badly centered, node distance=3cm, inner sep=0pt]
\tikzstyle{block} = [rectangle, draw, fill=blue!20, 
    text width=5em, text centered, rounded corners, minimum height=4em]
\tikzstyle{line} = [draw, -latex']
\tikzstyle{cloud} = [draw, ellipse,fill=red!20, node distance=3cm,
    minimum height=2em]

\begin{figure}
\centering
\begin{tikzpicture}[scale = 2, node distance = 2cm, auto]
    % Background/Collection boxes
    \node[rectangle, draw, minimum width=6cm, minimum height=4cm, fill=green!10]
      (r) at (4.75,-2.5) {};
    \node[rectangle, draw, minimum width=6cm, minimum height=4.5cm, fill=yellow!10]
      (r) at (4.75,-5) {};
    % Place nodes
    \node [block] (init) {Start};
    \node [block, below of=init]       (read_data)  {Read data}; 
    \node [block, below of=read_data]  (split_data) {Split data};

    \node [cloud, right of=split_data, node distance=4cm] (repeat) {200 Epochs};
    
    \node [block, right of=repeat, node distance=4cm] (apply_nn) {Apply NN};
    \node [block, right of=apply_nn, node distance=3cm]   (train_loss)
      {Training loss};
    \node [block, below of=train_loss] (back_prop)  {Back Propagation};
    \node [block, left of=back_prop, node distance=3cm]   (opt_step)   
      {Optimizer step};
    
    \node [block, below of=opt_step, node distance=3cm] (apply_nn2) {Apply NN};
    \node [block, right of=apply_nn2, node distance=3cm]   (valid_loss)
      {Validation loss};
    \node [block, below of=valid_loss] (accuracy)  {Compute accuracy};
    \node [block, left of=accuracy, node distance=3cm]   (efficiency)   
      {Compute efficiency};
    
    \node [decision, below of=efficiency] (improve) {Did NN improve?};
    \node [block, below of=improve, node distance=3cm] (store) {Store new best weights};

    %\node [block] (init) {initialize modelasd};
    %\node [cloud, left of=init] (expert) {expert};
    %\node [cloud, right of=init] (system) {system};
    %\node [block, below of=init] (identify) {identify candidate models};
    %\node [block, below of=identify] (evaluate) {evaluate candidate models};
    %\node [block, left of=evaluate, node distance=3cm] (update) {update model};
    %\node [decision, below of=evaluate] (decide) {is best candidate better?};
    %\node [block, below of=decide, node distance=3cm] (stop) {stop};

    % Draw edges
    \path [line] (init)       -- (read_data);
    \path [line] (read_data)  -- (split_data);
    \path [line] (split_data) -- (repeat);
    \path [line] (repeat) -- (apply_nn);

    \path [line] (apply_nn)   -- (train_loss);
    \path [line] (train_loss)  -- (back_prop);
    \path [line] (back_prop)  -- (opt_step);

    \path [line] (opt_step) -- (apply_nn2);
    \path [line] (apply_nn2) -- (valid_loss);
    \path [line] (valid_loss) -- (accuracy);
    \path [line] (accuracy) -- (efficiency);

    \path [line] (efficiency) -- (improve);
    \path [line] (improve) -- node {yes} (store);

    \path[line] (improve) -| node {no} (repeat);
    \path[line] (store) -| (repeat);


    %\path [line] (init) -- (identify);
    %\path [line] (identify) -- (evaluate);
    %\path [line] (evaluate) -- (decide);
    %\path [line] (decide) -| node [near start] {yes} (update);
    %\path [line] (update) |- (identify);
    %\path [line] (decide) -- node {no}(stop);
    %\path [line,dashed] (expert) -- (init);
    %\path [line,dashed] (system) -- (init);
    %\path [line,dashed] (system) |- (evaluate);
\end{tikzpicture}
\caption{Main workflow}
\label{fig:main_workflow}
\end{figure}


% Explain main workflow
In \autoref{fig:main_workflow} the main workflow is shown. It starts by reading
the samples and continues by making a 80/20 split. Resulting in 80\% of the
samples being used for training and 20\% being used for validation.

After that we have two main blocks: The training and the validation. In case we
improve our NN, we save the state and repeat this cycle 200 times. Each
iteration is called an epoch.

Note that in my thesis we only store the weights of the neural network. Saving
other weights like the ones from the optimizer would enable us to continue
training later on and thus basically letting us to add new samples. Since this
wasn't needed, I didn't implement it.

\subsection{Testing Procedure}
% Explain current situation:
%  - Release when 1/2month
Current searches operate at an false alarm rate (FAR) of one per 2 month
(CITATION NEEDED, see 26). For a new approach to be considered for production
use, it has to meet or exceed this. While the original papers (citation needed)
operated with false-alarm probability only, this isn't sufficient because this
isn't a good metric when testing on a continuous data stream. This is because
FAP assumes discrete data.

FAP basically is a trehshold at which a given possible detection is considered
a true positive while FAR is 

\[
  \text{FAR} = \frac{N_f}{T_0}
\]

whereas $N_f$ is the number of false positives and $T_0$ is the duration of the
analyzed data. The number of false positives $N_f$ is determined by a fixed
FAP acting as a threshold.


%  - Original papers use FAP, not good. FAP vs. FAR/disc. vs. cont.
%  - Mention FAR, sensitivity from paper 26

% Explain FAR

% Explain FAP (which is needed for FAR)

% Explain sensitivity

\begin{itemize}
  \item Problem discrete data: Paper 48
  \item Standardized metrics (FAR, sensitivity): Paper 26
  \item Explain state of the art needs 1 in 2 months, far of 10-4 etc.
\end{itemize}
Paper 48 says that we can't use FAP etc.

Paper 26 standardizes the metrics: FAR and sensitivity

% Explain training
\subsection{Training}
\begin{itemize}
  \item Fixed interval training.
  \item Apply Network
  \item Apply loss function
  \item Back propagation
  \item Optimizer step
\end{itemize}

% Explain validation
\subsection{Validation}
\begin{itemize}
  \item Apply Network
  \item Apply loss function
  \item Compute efficiency 
  \item Compute accuracy
\end{itemize}

\subsection{Network}
In this thesis the neural network from \cite{schaefer2021training} is used.
TODO: FAR like used in gabbart et al is "wrong", so use ml strategies approach.

\subsection{Loss Function}

\subsection{Accuracy \& Efficiency}





