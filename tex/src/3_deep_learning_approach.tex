%!TEX root = ../thesis.tex

\section{Deep Learning Approach}
% Introduction: High-level overview about general idea behind DL.
% Main Workflow: Describe the main workflow (train + eval in each epoch)
% Training: Describe training
% Evaluation: Describe evaulation (incl. accuracy and efficiency)

% Quick introduction to deep learning
Deep learning is a machine learning approach which consists of processing
units, so called neurons, which are arranged in an array. Such an array makes up
a layer. One to several such layers make up a neural network (NN). Each neuron
acts like a filter extracting higher-level features from the raw input. On a 
basic level, a neural network is trained by repeatedly applying the training
data and comparing its results with the known labels.

% Extend on the introduction
For the DL approach one needs a training set and a test set. The training set
is split whereas 80\% is used for the actual trianing of the NN and 20\% is used
to evaluate the trained NN in each epoch. 

% Provide pseudo code for main workflow
As we can see in pseudocode X, we havet his workflow. yada yada yada

% Provide pseudo code for train()

% Provide pseudo code for validation()



In this thesis the neural network from \cite{schaefer2021training} is used.
TODO: FAR like used in gabbart et al is "wrong", so use ml strategies approach.



