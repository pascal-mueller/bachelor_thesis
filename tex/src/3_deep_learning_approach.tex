%!TEX root = ../thesis.tex

\section{Deep Learning Approach}
% Intro: Explain basic workflow of DL. Introduce all the little parts.
% Explain:
% - Loss function
% - Optimizer (basically just stated parameters)
% - Efficiency
% - Accuracy
% - Neural Network used
% - Training strategy used
% - USR

% Quick introduction to deep learning
Deep learning is a machine learning approach which consists of processing
units, so called \textit{neurons}, which are arranged in an array. Such an array 
makes up a \textit{layer}. One to several such layers make up a
\textit{neural network} (NN). Each neuron
acts like a filter extracting higher-level features from the raw input. (ref)

\autoref{fig:3_main_workflow} depicts a high level view of the deep learning approach workflow.
It starts by reading the $200'000$ samples which were generated as described in
Chapter 2.1. After that, we make a 80/20 split whereas 80\% will make up the
training set and 20\% will make up the validation set.
After that, we iterate over the training and validation set for 200 epochs. In
each epoch we have a training section and a validation section. While they seem
similar, they are in fact very different which will be explains later on.

In a classical deep learning algorithm, you need a loss function and some
training progress validation metrics. We use

- accuracy and

- efficiency.

blabla which is why we first explain those.


%!TEX root = ../thesis.tex

% Define block styles
\tikzstyle{decision} = [diamond, draw, fill=blue!20, 
    text width=4.5em, text badly centered, node distance=3cm, inner sep=0pt]
\tikzstyle{block} = [rectangle, draw, fill=blue!20, 
    text width=5em, text centered, rounded corners, minimum height=4em]
\tikzstyle{line} = [draw, -latex']
\tikzstyle{cloud} = [draw, ellipse,fill=red!20, node distance=3cm,
    minimum height=2em]

\begin{figure}
\centering
\begin{tikzpicture}[scale = 2, node distance = 2cm, auto]
    % Background/Collection boxes
    \node[rectangle, draw, minimum width=6cm, minimum height=4cm, fill=green!10]
      (r) at (4.75,-2.5) {};
    \node[rectangle, draw, minimum width=6cm, minimum height=4.5cm, fill=yellow!10]
      (r) at (4.75,-5) {};
    % Place nodes
    \node [block] (init) {Start};
    \node [block, below of=init]       (read_data)  {Read data}; 
    \node [block, below of=read_data]  (split_data) {Split data};

    \node [cloud, right of=split_data, node distance=4cm] (repeat) {200 Epochs};
    
    \node [block, right of=repeat, node distance=4cm] (apply_nn) {Apply NN};
    \node [block, right of=apply_nn, node distance=3cm]   (train_loss)
      {Training loss};
    \node [block, below of=train_loss] (back_prop)  {Back Propagation};
    \node [block, left of=back_prop, node distance=3cm]   (opt_step)   
      {Optimizer step};
    
    \node [block, below of=opt_step, node distance=3cm] (apply_nn2) {Apply NN};
    \node [block, right of=apply_nn2, node distance=3cm]   (valid_loss)
      {Validation loss};
    \node [block, below of=valid_loss] (accuracy)  {Compute accuracy};
    \node [block, left of=accuracy, node distance=3cm]   (efficiency)   
      {Compute efficiency};
    
    \node [decision, below of=efficiency] (improve) {Did NN improve?};
    \node [block, below of=improve, node distance=3cm] (store) {Store new best weights};

    %\node [block] (init) {initialize modelasd};
    %\node [cloud, left of=init] (expert) {expert};
    %\node [cloud, right of=init] (system) {system};
    %\node [block, below of=init] (identify) {identify candidate models};
    %\node [block, below of=identify] (evaluate) {evaluate candidate models};
    %\node [block, left of=evaluate, node distance=3cm] (update) {update model};
    %\node [decision, below of=evaluate] (decide) {is best candidate better?};
    %\node [block, below of=decide, node distance=3cm] (stop) {stop};

    % Draw edges
    \path [line] (init)       -- (read_data);
    \path [line] (read_data)  -- (split_data);
    \path [line] (split_data) -- (repeat);
    \path [line] (repeat) -- (apply_nn);

    \path [line] (apply_nn)   -- (train_loss);
    \path [line] (train_loss)  -- (back_prop);
    \path [line] (back_prop)  -- (opt_step);

    \path [line] (opt_step) -- (apply_nn2);
    \path [line] (apply_nn2) -- (valid_loss);
    \path [line] (valid_loss) -- (accuracy);
    \path [line] (accuracy) -- (efficiency);

    \path [line] (efficiency) -- (improve);
    \path [line] (improve) -- node {yes} (store);

    \path[line] (improve) -| node {no} (repeat);
    \path[line] (store) -| (repeat);


    %\path [line] (init) -- (identify);
    %\path [line] (identify) -- (evaluate);
    %\path [line] (evaluate) -- (decide);
    %\path [line] (decide) -| node [near start] {yes} (update);
    %\path [line] (update) |- (identify);
    %\path [line] (decide) -- node {no}(stop);
    %\path [line,dashed] (expert) -- (init);
    %\path [line,dashed] (system) -- (init);
    %\path [line,dashed] (system) |- (evaluate);
\end{tikzpicture}
\caption{Main workflow}
\label{fig:main_workflow}
\end{figure}


\newpage

\subsection{The Neural Network}
The neural network used in this work was taken from [ML strategies]. It consists
of 6 convolutional layers followed by two linear layers which each include
dropouts as regularization. While [ml strategies] randomizes the initial weights
, this isn't done in this work because for one, we don't average any metrics and
while working with it, it always converged fast and gave good results. I like to
keep things simple. <Repeat arguments from ml strategies paper about why we
chose some parameters like 2048 Hz>

% Introduction to DL
\begin{tabular}{ |p{5cm}||l| }
  \hline
  \multicolumn{2}{|c|}{Neural Network} \\
  \hline
  Type & Parameters \\
  \hline
  Batch Normalization & \\
  Conv1d + ELU      & $C_\text{in}=1$, $C_\text{out}=8$, Kernel size = 64 \\
  
  Conv1d            & $C_\text{in}=8$, $C_\text{out}=8$, Kernel size = 32 \\
  MaxPool1d + ELU   & Kernel size = 4 \\
  
  Conv1d + ELU      & $C_\text{in}=8$, $C_\text{out}=16$, Kernel size = 32 \\
  
  Conv1d            & $C_\text{in}=16$, $C_\text{out}=16$, Kernel size = 16 \\
  MaxPool1d + ELU   & Kernel size = 3 \\
  
  Conv1d + ELU      & $C_\text{in}=16$, $C_\text{out}=32$, Kernel size = 16 \\
  
  Conv1d            & $C_\text{in}=32$, $C_\text{out}=32$, Kernel size = 16 \\
  MaxPool1d + ELU   & Kernel size = 2 \\

  Flatten           & \\

  Linear            & in\_features = 1856, out\_features = 64 \\
  Dropout + ELU     & $p=0.8$ \\

  Linear            & in\_features = 64, out\_features = 64 \\
  Dropout + ELU     & $p=0.8$ \\

  Linear            & in\_features = 64, out\_features = 2 \\

  Softmax and Linear           & \\
  \hline
\end{tabular}

Note that the output layer consists of a softmax and a linear layer, which mean
that we feed the logits from the last hidden layer to both of them and return
two predictions. TODO: add bijective stuff.

\subsection{Loss Function}
The loss function was chosen as a varian of the binary cross-entropy that is
designed to stay finite [ml strategies]

\begin{equation}
  L(\mathbf{y}_t, \mathbf{y}_p) = -\frac{1}{N_b} \sum_{i=1}^{N_b}
  \mathbf{y}_{t,i} \cdot \log(\epsilon + (1 - 2\epsilon) \mathbf{y}_{p,i})
\end{equation}

whereas $\mathbf{y}_p = (1, 0)^T$ if the sample contained a signal or
$\mathbf{y}_p = (0, 1)^T$ if the sample contained pure noise. $\mathbf{y}_p$ is
the prediction of the network for the sample. TODO: add Nb and eps.

The implementation of this loss function can be found in
\hlc{src/reg\_BCELoss.py} and was taken from [ml strategies].

\subsection{Accuracy \& Efficiency}

% Explain training
\subsection{Training}
The training loop, implemented in the \hlc{train()} function in the file
\hlc{train.py}, is depicted in sample code 1.

\begin{itemize}
  \item Fixed interval training.
  \item Apply Network
  \item Apply loss function
  \item Back propagation
  \item Optimizer step
\end{itemize}

% Explain validation
\subsection{Validation}
\begin{itemize}
  \item Apply Network
  \item Apply loss function
  \item Compute efficiency 
  \item Compute accuracy
\end{itemize}







\subsection{Testing Procedure}
% Explain current situation:
%  - Release when 1/2month
Current searches operate at an false alarm rate (FAR) of one per 2 month
(CITATION NEEDED, see 26). For a new approach to be considered for production
use, it has to meet or exceed this. While the original papers (citation needed)
operated with false-alarm probability only, this isn't sufficient because this
isn't a good metric when testing on a continuous data stream. This is because
FAP assumes discrete data.

FAP basically is a trehshold at which a given possible detection is considered
a true positive while FAR is 

\[
  \text{FAR} = \frac{N_f}{T_0}
\]

whereas $N_f$ is the number of false positives and $T_0$ is the duration of the
analyzed data. The number of false positives $N_f$ is determined by a fixed
FAP acting as a threshold.


%  - Original papers use FAP, not good. FAP vs. FAR/disc. vs. cont.
%  - Mention FAR, sensitivity from paper 26

% Explain FAR

% Explain FAP (which is needed for FAR)

% Explain sensitivity

\begin{itemize}
  \item Problem discrete data: Paper 48
  \item Standardized metrics (FAR, sensitivity): Paper 26
  \item Explain state of the art needs 1 in 2 months, far of 10-4 etc.
\end{itemize}
Paper 48 says that we can't use FAP etc.

Paper 26 standardizes the metrics: FAR and sensitivity

TODO:
\begin{itemize}
  \item Improve NN table
  \item Add references
\end{itemize}
